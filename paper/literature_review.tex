\section{Literature Review}

% 1. mothers parental leave results and how they affect male parental leave
The literature of parental leave policy research focusses mostly on mothers' labor market outcomes, income, and child well-being.
As the take-up of parental leave by mothers affects strongly the behavior of fathers \citep[see][]{sigurdardottirBacklashGenderEquality2018, canaanMaternityLeavePaternity2022}%reference here more
it is important to investigate both to understand the full effects of parental leave policies on fathers income.
\citet{baker2008does} investigate two Canadian maternity leave expansions that lead to a higher job continuity for mothers but also find that the
extended time of leave is mostly taken up by mothers.
\citet{laliveParentalLeaveMothers2014} finds similar results in Austria where an extension in maternity leave time leads to strongly increased leave durations
while take-up for fathers remains low.
 \citet{schonbergExpansionsMaternityLeave2014} investigates multiple German parental leave reforms and finds that mothers spend more time with children with longer
maternity leave durations but does not find any improvements for mothers income or labor market attachments 2 to 6 years after birth.
Another study in the German context investigates the 2007 \textit{Elterngeld} reform heterogenous effect on different income groups of mother. The author find that
low-income mothers fertility rates decreased while that of high income mothers increased, suggesting a more hostile environment to bear children for low-income mothers
\citep{cygan-rehmParentalLeaveBenefit2016}.
\citet{frodermannParentalLeavePolicy2023} investigate the same intervention and find positive long-run effect on earnings for mothers with high pre-birth earnings but not for low-income mothers.
They add that the introduction of the "daddy months" within this reform facilitates more involvemenet of fathers and thus increased earnings for at least
high income mothers.
Parental leave policies heterrogoneous effects across different socioeconomic groups seem to be robust across different high income countries \citep{canaanMaternityLeavePaternity2022}.
Though the authors find beneficial effects for mothers and childrens well-being. \\
The results of these studies suggest that there is an adverse effect of parental leave on mothers' labor market outcomes and income. Namely,
that an extension of parental leave that allocates more time or ressources to the couple or solely to the mother might negatively effect mothers
household decision making as the opportunity cost of taking care of the child become higher for fathers.
\citet{canaanMaternityLeavePaternity2022} argue that activating fathers to take parental leave might diminsh these adverse effects.
%1.1state some facts about findings that have impact on equality and their partners decision makeing

This suggests introducing or extending parental leave for fathers could have beneficiary effects for mothers and thus lead to more equality but the research is scarce.
\citet{patnaikReservingTimeDaddy2019} investigates a paternity leave program in Quebec, Canada and finds that fathers that take-up parentel leave spend more time in housework and with childcare.
\citet{tammFathersParentalLeaveTaking2018} finds that the \textit{Elterngeld} reform in Germany leads to a reduction of gender differenes in housework and stable labor hours in the household.
Suggesting that fathers reduction in labor hours is compensated by the mother.
Investigating the same policy \citet{frodermannParentalLeavePolicy2023} support those results as they find positive effects on mothers' earnings due to fathers' involvement in child care and house-work.
These results indicated that fathers' take-up of parental leave has positive effects regarding household equality and to a lesser extend on mothers' labor market outcomes.
The question remains if fathers take-up of parental leave diminishes adverse effects for mothers, why is
the take-up rates seem generally low \citep[see][]{laliveParentalLeaveMothers2014, frodermannParentalLeavePolicy2023}.

A possible reason are the negative effects parental leave policies can have for fathers, especially regarding their income.
\citet{jorgensenWelfareReformsDivision2021} underline the importance of wage replacement rates in influencing the uptake of parental leave.
\citet{regeImpactPaternityLeave2013} investigates the introduction of a father quota in Norway and finds that fathers that take-up of parental leave reduces significantly their income.
A study similar to this thesis inestigates the effect of the \textit{Elterngeld} reform on fathers labor hours finds reduced working hours for fathers that take parental leave suggesting lower income \citep{bunningWhatHappensDaddy2015}.
There are few results regarding negative effects of parental leave on fathers income making the research on this topic important.
