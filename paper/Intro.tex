\section{Introduction}

% 1.1 inequality and their importance in society



% 1.2. how parental leave can affect inequality
% speak about the effect on women and how it affects men and the HH decision making
% focus on the adverse effect of parental leave for mothers (i.e. why it is not only well-beng increase but also maybe enhances inequality)
Most literature focusses on parental leave of mothers and its impact on labor market outcomes, well-being, and income of mothers.
This seems comprehensible due to the fact that most countries have implemented parental leave policies focussing mothers.
Parental Leave policies for fathers are less common, often shorter and the take-up rates are generally lower (see cite here) thus harder to investigate for researchers.
While parental leave for mothers is

% 1.3 how parental leave for men affects the equality discussion

% 1.4. the contribution of this paper
First, the paper contributes to the literature by examining the effect of parental leave on fathers income.
Second, I introduce a more robust estimation approach for Difference in Difference (DiD) models, which can be considered the standard in parental leave policy investigations.
Finally, I provide some comments about the behavior and perfomance of semiparemtric estimators in the small sample case.

% 1.5. structure of the paper
